After demonstrating that the PIGSFLI algorithms operates as expected according to theory, I was given the opportunity to continue working with the DelMaestro group virtually through their weekly Zoom research meetings. During the time I worked with the group virtually, I profiled the PIGSFLI algorithm to find where it was spending more time than necessary. Although this search was not fully successful, I learned much about both how the code operates and many aspects of C++ which I had not yet learned. In the Fall of 2024, I will be attending the University of Tennessee Knoxville for their P.h.D. program in condensed matter theory, where I will continue to work with the DelMaestro research group. Not only will I most likely continue work with the PIGSFLI code and the Bose-Hubbard model, but I may also act as a teaching assistant for Dr. Del Maestro's Introduction to Machine Learning for Scientists. I look forward in continuing work with the DelMaestro group on the Bose-Hubbard model and machine learning.
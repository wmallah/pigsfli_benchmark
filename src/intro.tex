% Tell the story
%% Talk about what quantum computing is and why it is important
%% Funnel towards the work that I did this past summer
%% Mainly discuss entanglement entropy and why it is important


% What is quantum computing?
%% Give dates
Richard Feynman was the first to conceptualize the idea of using a quantum mechanical system to perform calculations. Not only would this machine be able to perform calculations, but it would also be capable of simulating physical quantum mechanical problems. Through further analysis, Feynman found that quantum computers could solve problems which could not be solved by classical computers, as the calculations may last longer than the age of the universe. More specifically, classical computers require exponentially growing time to solve quantum mechanical many body problems as systems become more complex (i.e., more particles), whereas quantum computers could do so in polynomial time. It was then later discovered by David Deutsch of Oxford University, a man whose name is sprinkled through many quantum computing methods and algorithms, that a general-purpose quantum computer is theoretically possible. Moreover, he ``showed that any physical process, in principle could be modelled perfectly by a quantum computer'' \cite{singh_study_2005}.

The first major application of quantum computation was discovered by Peter Shor, famous for Shor's factorization algorithm: this algorithm was used to successfully factor huge numbers rapidly. Since all information on classical computers is kept safe through encryption methods of factoring large prime numbers, this breakthrough gained the attention of many around the world. Now, with quantum computation on the rise, people in fields external to physics and information science have begun to discover new applications. One such important application is simulating molecules, also known as quantum chemistry. Simulating quantum chemistry will aid the medical field in a revolutionary way: possibly leading to a cure for cancer or other diseases/ailments. Similarly, quantum computation may be used to eradicate starvation through optimal allocation of resources to people around the world. Similar to a quantum computer's ability to simulate large quantum mechanical system sizes where there are enumerable possible states the system can exist in, quantum computers are efficient at running optimization algorithms where there are very large numbers of states for the system. For this reason, quantum computation would be useful to distribute resources around the globe. These applications demonstrate the possibility for the utopian reality we've all read and dreamt about: a disease and hunger-free world.

% What is entanglement entropy?
One measurement that is of particular interest in quantum computation is entanglement entropy. Entanglement is a strictly quantum phenomenon where particles within a system cannot be represented mathematically in a way that is completely separate from each other: in short, the particles are fundamentally connected to each other. A more general way to quantify this correlation between partitions of a system is entanglement entropy. Entanglement entropy generalizes from single particles to subsystems. Explicitly, entanglement entropy quantifies the correlations between partitions of a system. Like regular entanglement, entanglement entropy rises from the partitions of the system lacking definite quantum states on their own: i.e., the state of each partition depends on the other. The most general definition is that ``Entanglement entropy is a measure of how quantum information is stored in a quantum state'' \cite{hartman_lectures_2015}. The entanglement piece stems from the uncertainty in the state of the system as a result of the entanglement between partitions. This concept is very similar to classical entropy where the microstate of the system is uncertain. This measure of entanglement entropy was the focus of the research I conducted over the summer of 2023 and continued during the 2023-2024 school year.

% Introduction to the work I did over the summer
During the summer of 2023, I worked with Dr. Adrian Del Maestro and his research group at the University of Tennessee Knoxville while attending one of their Research Experience for Undergraduates (REU) programs titled Quantum Algorithms and Optimization (QAO). During this experience, I worked on benchmarking their algorithm for simulating identical, interacting bosons on a lattice. Specifically, I worked with the small scale (2 particles on 2 sites) Bose-Hubbard model in order to confidently extend the simulation to a higher scale (larger system sizes). I began my work by studying the theory of the Bose-Hubbard model (\cref*{calculations}). After learning the theory, I began simulating the system using the PIGSFLI algorithm (\cref*{PIGSFLI}). I then compared the results of the PIGSFLI algorithm to the exact diagonalization (ED) results (\cref*{results}). 
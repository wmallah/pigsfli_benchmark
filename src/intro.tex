% Tell the story
%% Talk about what quantum computing is and why it is important
%% Funnel towards the work that I did this past summer
%% Mainly discuss entanglement entropy and why it is important


% What is quantum computing?
%% Give dates
Richard Feynman was the first to conceptualize the idea of using a quantum mechanical system to perform calculations. Not only would this machine be able to perform calculation, but it would also be capable of simulating physical quantum mechanical problems. Through further analysis, Feynman found that quantum computers could solve problems which could not be solved by classical computers, as the calculations may last longer than the age of the universe. More specifically, classical computers require exponentially growing time to solve quantum mechanical many body problems, whereas quantum computers could do so in polynomial time. It was then later discovered by Deutsch, a man whose name is sprinkled through many quantum computing methods and algorithms, that a general-purpose quantum computer is theoretically possible. Moreover, he "showed that any physical process, in principle could be modelled perfectly by a quantum computer" \cite{singh_study_2005}.

The first major application of quantum computation was discovered by Peter Shor, famous for Shor's factorization algorithm: this algorithm was used to successfully factor huge numbers rapidly. Since all information on classical computers is kept safe through encryption methods of factoring large prime numbers, this breakthrough gained the attention of many around the world.

% What is entanglement entropy?
One measurement that is of particular interest in quantum computation is entanglement entropy: "Entanglement entropy is a measure of how quantum information is stored in a quantum state" \cite{hartman_lectures_2015}.

% Introduction to the work I did over the summer
During the summer of 2023, I worked with Dr. Adrian Del Maestro and his research group at the University of Tennessee Knoxville while attending one of their Research Experience for Undergradutes (REU) programs titled Quantum Alogorithms and Optimization (QAO). During this experience, I worked on benchmarking their algorithm for simulating identical, interacting bosons on a lattice. More specifically, I worked with the small scale (2 particles on 2 sites) in order to confidently extend the simulation to higher scale (larger system sizes). I began my work by studying the theory of the Bose-Hubbard model (\cref*{calculations}). After learning the theory, I began simulating the system using the PIGSFLI algorithm (\cref*{PIGSFLI}). I then compared the results of the PIGSFLI algorithmn to the exact diagonalization (ED) results (\cref*{results}). 


% Finally, I used machine learning to predict the entanglement entropy of the system (\cref*{machine_learning}).

% Give layout of thesis